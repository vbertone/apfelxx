%% LyX 2.0.3 created this file.  For more info, see http://www.lyx.org/.
%% Do not edit unless you really know what you are doing.
\documentclass[twoside,english]{paper}
\usepackage{lmodern}
\renewcommand{\ttdefault}{lmodern}
\usepackage[T1]{fontenc}
\usepackage[latin9]{inputenc}
\usepackage[a4paper]{geometry}
\geometry{verbose,tmargin=3cm,bmargin=2.5cm,lmargin=2cm,rmargin=2cm}
\usepackage{color}
\usepackage{babel}
\usepackage{float}
\usepackage{bm}
\usepackage{amsthm}
\usepackage{amsmath}
\usepackage{amssymb}
\usepackage{graphicx}
\usepackage{esint}
\usepackage[unicode=true,pdfusetitle,
 bookmarks=true,bookmarksnumbered=false,bookmarksopen=false,
 breaklinks=false,pdfborder={0 0 0},backref=false,colorlinks=false]
 {hyperref}
\usepackage{breakurl}
\usepackage{makeidx}

\makeatletter

%%%%%%%%%%%%%%%%%%%%%%%%%%%%%% LyX specific LaTeX commands.
%% Because html converters don't know tabularnewline
\providecommand{\tabularnewline}{\\}

%%%%%%%%%%%%%%%%%%%%%%%%%%%%%% Textclass specific LaTeX commands.
\numberwithin{equation}{section}
\numberwithin{figure}{section}

%%%%%%%%%%%%%%%%%%%%%%%%%%%%%% User specified LaTeX commands.
\usepackage{babel}

\@ifundefined{showcaptionsetup}{}{%
 \PassOptionsToPackage{caption=false}{subfig}}
\usepackage{subfig}
\makeatother

\usepackage{listings}


\begin{document}

\title{Generalised parton distributions}

\author{Valerio Bertone}

\tableofcontents{}

\section{Introduction}

In this set of notes I collect the technical aspects concerning
generalised parton distributions (GPDs). Since the computation GPDs
introduces new kinds of convolution integrals, a strategy aimed at
optimising the numerics needs to be devised.

\section{Evolution equation}

The evolution equation for GPDs reads:\footnote{It should be noticed
  that the integration bounds of the integration in
  Eq.~(\ref{eq:eveq}) are dictated by the operator defintion of the
  distribution $f$ on the light cone and not by the kernel
  $\mathbb{V}$.}
\begin{equation}\label{eq:eveq}
\mu^2\frac{d}{d\mu^2}f(x,\xi) = \int_{-1}^{1}\frac{dx'}{\left|2\xi\right|}\mathbb{V}\left(\frac{x}{\xi},\frac{x'}{\xi}\right)f(x',\xi)\,.
\end{equation}
In general, the GPD $f$ and the evolution kernel $\mathbb{V}$ should
be respectively interpreted as a vector and a matrix in flavour
space. However, for now, we will just be concerned with the integral
in the r.h.s. of Eq.~(\ref{eq:eveq}) regardless of the flavour
structure.

The support of the evolution kernel
$\mathbb{V}\left(\frac{x}{\xi},\frac{x'}{\xi}\right)$ is shown in
Fig.~\ref{fig:GPDIntDomain}.
\begin{figure}[h]
  \begin{centering}
    \includegraphics[width=0.7\textwidth]{plots/GPDIntDomain}
    \caption{Support domain of the evolution kernel
      $\mathbb{V}\left(\frac{x}{\xi},\frac{x'}{\xi}\right)$.\label{fig:GPDIntDomain}}
  \end{centering}
\end{figure}
The Knowledge of the support region of the evolution kernel allows us
to rearrange Eq.~(\ref{eq:eveq}) as follows:
\begin{equation}
\displaystyle\mu^2\frac{d}{d\mu^2}f(\pm x,\xi) =\int_{b(x)}^{1}\frac{dx'}{x'}\left[\frac{x'}{\left|2\xi\right|}\mathbb{V}\left(\pm \frac{x}{\xi},\frac{x'}{\xi}\right)f(x',\xi)+\frac{x'}{\left|2\xi\right|}\mathbb{V}\left(\mp \frac{x}{\xi},\frac{x'}{\xi}\right)f(-x',\xi)\right]\,,
\end{equation}
with:
\begin{equation}\label{eq:lowintb}
b(x) = |x|\theta\left(\left|\frac{x}{\xi}\right|-1\right)\,,
\end{equation}
and where we have used the symmetry property of the evolution kernels:
$\mathbb{V}(y,y')=\mathbb{V}(-y,-y')$. In the unpolarised case, it is
useful to define:\footnote{Notice the seemingly unusual fact that
  $f^{+}$ is defined as difference and $f^{-}$ as sum of GPDs computed
  at opposite values of $x$. This can be understood from the fact
  that, in the forward limit, $f(-x)= -\overline{f}(x)$, \textit{i.e.}
  the PDF of a quark computed at $-x$ equals the PDF of the
  corresponding antiquark computed at $x$ with opposite sign. The
  opposite sign is absent in the longitudinally polarised case.}
\begin{equation}\label{eq:pmdef}
\begin{array}{rcl}
\displaystyle f^{\pm}(x,\xi) &=&\displaystyle  f(x,\xi) \mp
                       f(-x,\xi)\,,\\
\\
\displaystyle \mathbb{V}^{\pm}(y,y') &=&\displaystyle  \mathbb{V}(y,y') \mp \mathbb{V}(-y,y')\,,
\end{array}
\end{equation}
so that the evolution equation for $f^{\pm}$ reads:
\begin{equation}\label{eq:eveq2}
\displaystyle\mu^2\frac{d}{d\mu^2}f^{\pm}(x,\xi) = \int_{b(x)}^{1}\frac{dx'}{x'}\frac{x'}{\left|2\xi\right|}
                                                         \mathbb{V}^{\pm}\left(\frac{x}{\xi},\frac{x'}{\xi}\right)f^{\pm}(x',\xi)\,.
\end{equation}
The $f^{\pm}$ distributions can be regarded as the GPD analogous of
the $\pm$ forward distributions that can then be used to construct the
usual singlet and non-singlet distributions in the QCD evolution
basis. This uniquely determines the flavour structure of the evolution
kernels $\mathbb{V}^{\pm}$.

It is relevant to observe that the presence of the $\theta$-function
in the lower integration bound $b$, Eq.~(\ref{eq:lowintb}), is such
that for $|x|>|\xi|$ the evolution equation has the exact form of the
DGLAP evolution equation which corresponds to integrating over the
blue regions in Fig.~\ref{fig:GPDIntDomain} (DGLAP region,
henceforth). Conversely, for $|x|\leq|\xi|$ the lower integration
bound becomes zero and the evolution equation assumes the form of the
so-called ERBL equation that describes the evolution of meson
distribution amplitudes (DAs). This corresponds to integrating over
the red region (ERBL region, henceforth). Crucially, in the limits
$\xi\rightarrow 0$ and $\xi\rightarrow \pm1$ Eq.~(\ref{eq:eveq2})
needs to recover the DGLAP and ERBL equations, respectively.

% GPD anomalous dimensions are generally tricky to integrate numerically
% because of the intricate support. In order to simplify the integration
% procedure, we can decompose the anomalous dimensions using the labels
% given in Fig.~\ref{fig:GPDIntDomain} as a guide:
% \begin{equation}
% \begin{array}{rcl}
% \displaystyle\mathbb{V}(y,y')&=&\displaystyle
%   \theta(y')\\
% \\
% &\times&\Big[\theta(y-1)\theta(y'-y)\mathbb{V}_A(y,y')+\theta(1-y)
%   \theta(y'-y)\mathbb{V}_B(y,y') +\theta(1-y)
%   \theta(y-y')\mathbb{V}_C(y,y')\\
% \\
% &+&\displaystyle \theta(-y-1)\theta(y+y')\mathbb{V}_{\overline{A}}(y,y')+\theta(1+y)
%   \theta(y+y')\mathbb{V}_{\overline{B}} (y,y') +\theta(1+y)
%   \theta(-y'-y)\mathbb{V}_{\overline{C}} (y,y')\Big]\\
% \\
% &+&\displaystyle \theta(-y')\\
% \\
% &\times&\displaystyle\Big[\theta(y-1)\theta(-y-y')\mathbb{V}_{\overline{A}}(y,y')+\theta(1-y)
%   \theta(-y-y')\mathbb{V}_{\overline{B}} (y,y') +\theta(1-y)
%   \theta(y'+y)\mathbb{V}_{\overline{C}} (y,y')\\
% \\
% &+&\displaystyle\theta(-y-1)\theta(-y'+y)\mathbb{V}_A(y,y')+\theta(1+y)
%   \theta(-y'+y)\mathbb{V}_B(y,y') +\theta(1+y)
%   \theta(-y+y')\mathbb{V}_C(y,y')\Big]\,,
% \end{array}
% \end{equation}
% where the functions $\mathbb{V}_I$ and $\mathbb{V}_{\overline{I}}$,
% with $I=A,B,C$, are defined on the respective regions in
% Fig.~\ref{fig:GPDIntDomain}.\footnote{Note that $\mathbb{V}_I(y,y')$
%   and $\mathbb{V}_{\overline{I}}(y,y')$ are not required to be
%   symmetric upon the transformation
%   $(y \rightarrow -y, y' \rightarrow -y')$.}  Next, we take the
% combinations given in Eq.~(\ref{eq:pmdef}) relevant to implement the
% evolution equation in Eq.~(\ref{eq:eveq2}). By doing this, one
% obtains:
% \begin{equation}\label{eq:DGLAPsuitable}
%   \mathbb{V}^\pm(y,y')=\theta(y-1)\mathbb{V}_A^\pm(y,y')+\theta(1-y)
%   \left[\theta(y'-y)\mathbb{V}_B^\pm(y,y') +
%     \theta(y-y')\mathbb{V}_C^\pm(y,y')\right]\,,
% \end{equation}
% where we have defined:
% \begin{equation}\label{eq:pmdef}
% \mathbb{V}_I^{\pm}(y,y') = \mathbb{V}_I(y,y')\mp
% \mathbb{V}_{\overline{I}}(-y,y')\,,
% \end{equation}
% and omitted all the irrelevant/redundant terms and factors for the
% computation of the integral in the r.h.s. of
% Eq.~(\ref{eq:eveq2}). From Eq.~(\ref{eq:DGLAPsuitable}), it should be
% clear that the anomalous dimension $\mathbb{V}_A^\pm$ is responsible
% for the evolution in the DGLAP region while $\mathbb{V}_B^\pm$ and
% $\mathbb{V}_C^\pm$ are responsible for the evolution in the ERBL
% region. The latter observation suggests that $\mathbb{V}_B^\pm$ and
% $\mathbb{V}_C^\pm$ are related. The relation can easily be established
% by observing that the general structure of the ERBL anomalous
% dimensions is:
% \begin{equation}
% V^{\rm ERBL}(y,y')=\theta(y'-y)F(y,y')+\theta(y-y')F(-y,-y')\,,
% \end{equation}
% which immediately implies that:
% \begin{equation}
% \mathbb{V}_C^\pm(y,y')=\mathbb{V}_B^\pm(-y,-y')\,.
% \end{equation}
% Finally, one finds that a convenient decomposition for the anomalous
% dimension in Eq.~(\ref{eq:eveq2}) is:
% \begin{equation}
% \mathbb{V}^\pm(y,y')=\theta(y-1)\mathbb{V}_A^\pm(y,y')+\theta(1-y)
%   \left[\theta(y'-y)\mathbb{V}_B^\pm(y,y') +
%   \theta(y-y')\mathbb{V}_B^\pm(-y,-y')\right]\,.
% \end{equation}

% Eq.~(\ref{eq:eveq2}) can be further manipulated to make it resemble
% the structure of the DGLAP equation as much as possible. To this
% purpose, we define the parameter:
% \begin{equation}\label{eq:kappadef}
% \kappa(x) = \frac{\xi}{x}\,,
% \end{equation}
% so that:
% \begin{equation}\label{eq:manip}
% \frac{x'}{\left|2\xi\right|}
% \mathbb{V}_I^{\pm}\left(\pm\frac{x}{\xi}, \pm\frac{x'}{\xi}\right)={\rm sign}(\xi)\frac{1}{2\kappa}
% \frac{x'}{x} \mathbb{V}_I^{\pm}\left(\pm\frac{1}{\kappa}, \pm\frac{1}{\kappa}
%   \frac{x'}{x}\right)\equiv {\rm sign}(\xi)\mathcal{P}_I^{\pm}\left(\pm\kappa,\frac{x}{x'}\right)\,,
% \end{equation}
% where the last equality effectively defines the \textit{DGLAP-like}
% splitting function:
% \begin{equation}\label{eq:DGLAPevk}
%   \mathcal{P}_I^{\pm}(\pm\kappa,y) = \frac{1}{2\kappa y}
%   \mathbb{V}_I^{\pm}\left(\pm\frac{1}{\kappa}, \pm\frac{1}{\kappa y}\right)\,.
% \end{equation}
% In the following we will assume $\xi>0$ as, so far, this is the only
% experimentally accessible region. This allows us to get rid of
% ${\rm sign}(\xi)$ in Eq.~(\ref{eq:manip}). In addition, without loss
% of generality, we can also restrict ourselves to positive values of
% $x$ because negative values can be easily accessed by symmetry using
% Eq.~(\ref{eq:pmdef}), \textit{i.e.}
% $f^{\pm}(-x,\xi)=\mp f^{\pm}(x,\xi)$. Using the definition in
% Eq.~(\ref{eq:DGLAPevk}) in the integral in the r.h.s. of
% Eq.~(\ref{eq:eveq2}) and finally performing a change of variable
% gives:
% \begin{equation}\label{eq:DGLAPforGPDs}
% \displaystyle\mu^2\frac{d}{d\mu^2}f^{\pm}(x,\xi)= \int_{b(x)}^{1}\frac{dx'}{x'}\mathcal{P}^{\pm}\left(\kappa, \frac{x}{x'}\right)f^{\pm}\left(x',\xi\right)=\int_{x}^{x/b(x)}\frac{dy}{y}\mathcal{P}^{\pm}\left(\kappa,y\right)f^{\pm}\left(\frac{x}{y},\xi\right)\,,
% \end{equation}
% with:
% \begin{equation}
% b(x) = x\,\theta(1-\kappa)\,,
% \end{equation}
% and:
% \begin{equation}\label{eq:DGLAPevkdec}
%   \mathcal{P}^{\pm}\left(\kappa,y\right)=\theta(1-\kappa)\mathcal{P}_A^\pm(\kappa,y)+\theta(\kappa-1)
%   \left[\theta(1-y)\mathcal{P}_B^\pm(\kappa,y) +
%     \theta(y-1)\mathcal{P}_B^\pm(-\kappa,y)\right]\,.
% \end{equation}
% Plugging Eq.~(\ref{eq:DGLAPevkdec}) into Eq.~(\ref{eq:DGLAPforGPDs}),
% one obtains:
% \begin{equation}\label{eq:DGLAPforGPDs2}
% \begin{array}{rcl}
%   \displaystyle\mu^2\frac{d}{d\mu^2}f^{\pm}(x,\xi)&=&\displaystyle
%                                                       \theta(1-\kappa)\int_{x}^{1}\frac{dy}{y}\mathcal{P}_A^{\pm}\left(\kappa,y\right)f^{\pm}\left(\frac{x}{y},\xi\right)\\
%   \\
%                                                   &+&\displaystyle\theta(\kappa-1)\int_{x}^{\infty}\frac{dy}{y}\left[\theta(1-y)\mathcal{P}_B^{\pm}\left(\kappa,
%                                                       y\right)+\theta(y-1)\mathcal{P}_B^{\pm}\left(-\kappa,y\right)\right]f^{\pm}\left(\frac{x}{y},\xi\right)\,.
% \end{array}
% \end{equation}
% Eq.~(\ref{eq:DGLAPforGPDs2}) has almost the form of a ``standard''
% DGLAP equation except for the upper bound of the integral in the
% second line that extends up to infinity. However, this kind of
% integrals can be handled within APFEL with minor modifications of the
% integration strategy and up to a numerical approximation to be
% assessed.

% \subsection{On continuity of GPDs}

% It is well known that GPDs are required to be continuous at $x=\xi$
% for factorisation to be valid~\cite{Radyushkin:1997ki}. It is thus
% interesting to consider the consequence of this constraint. To this
% end, let us consider the limits of Eq.~(\ref{eq:DGLAPforGPDs2}) for
% $x\rightarrow \xi^\pm$, which corresponds to
% $\kappa\rightarrow 1^{\pm}$:
% \begin{equation}\label{eq:limit1}
% \lim_{x\rightarrow
%   \xi^+}\displaystyle\mu^2\frac{d}{d\mu^2}f^{\pm}(x,\xi) =
% \int_{x}^{1}\frac{dy}{y}\mathcal{P}_B^{\pm}\left(1,y\right)f^{\pm}\left(\frac{x}{y},\xi\right)+\int_{1}^{\infty}\frac{dy}{y}\mathcal{P}_B^{\pm}\left(-1,
%                                                       y\right)f^{\pm}\left(\frac{x}{y},\xi\right)\,,
% \end{equation}
% and:
% \begin{equation}\label{eq:limit2}
% \lim_{x\rightarrow \xi^-}\displaystyle\mu^2\frac{d}{d\mu^2}f^{\pm}(x,\xi) = \mu^2\frac{d}{d\mu^2}f^{\pm}(\xi,\xi)=\int_{x}^{1}\frac{dy}{y}\mathcal{P}_A^{\pm}\left(1,y\right)f^{\pm}\left(\frac{x}{y},\xi\right)\,.
% \end{equation}
% Taking the difference between Eqs.~(\ref{eq:limit1})
% and~(\ref{eq:limit2}), using the continuity of $f^{\rm}$ at $x=\xi$,
% and considering that:\footnote{We will prove this equality case by
%   case.}
% \begin{equation}\label{eq:continuity}
% \mathcal{P}_A^{\pm}\left(1,y\right) = \mathcal{P}_B^{\pm}\left(1,y\right)\,,
% \end{equation}
% one finds:
% \begin{equation}
% \int_{1}^{\infty}\frac{dy}{y}\mathcal{P}_B^{\pm}\left(-1,
%                                                       y\right)f^{\pm}\left(\frac{x}{y},\xi\right)=0\,,
% \end{equation}
% which has to be valid at any scale and for any $f^{\pm}$. This
% immediately implies that:
% \begin{equation}\label{eq:contcondition}
% \mathcal{P}_B^{\pm}\left(-1, y\right) = 0\,,
% \end{equation}
% for all values of $y$ and order-by-order in perturbation theory. We
% will explicitly verify this constraint when we will discuss the
% explicit expressions.

\subsection{End-point contributions}\label{sec:endpoint}

Some of the expressions for the anomalous dimensions discussed below
contain $+$-prescribed terms. It is thus important to treat these
terms properly. We are generally dealing with objects defined as:
\begin{equation}\label{eq:pludistributionn}
  \left[\mathbb{V}\left(x,x'\right)\right]_+=
  \mathbb{V}\left(x,x'\right)- \delta(x-x')\int_{-1}^{1}dx\,\mathbb{V}\left(x,x'\right)\,.
\end{equation}
where the function $\mathbb{V}$ has a pole at $x'=x$.

Let us take as an example the one-loop non-singlet anomalous
dimension. For definiteness, we will refer for the precise expression
to Eq.~(101) of Ref.~\cite{Diehl:2003ny} and report it here for
convenience (up to a factor $\alpha_s/4\pi$):
\begin{equation}\label{eq:diehlexpr}
V_{\rm NS}^{(0)}(x,x') = 2C_F\left[\rho(x,x')\left\{\frac{1+x}{1+x'}\left(1+\frac{2}{x'-x}\right)\right\}+(x\rightarrow -x, x'\rightarrow -x')\right]_+\,,
\end{equation}
with:\footnote{There is probably a typo in Eq.~(102) of
  Ref.~\cite{Diehl:2003ny} as the second $-1$ should actually be a
  $+1$.}
\begin{equation}\label{eq:supportdiehl}
\rho(x,x')=\theta(-x + x')\theta(1 + x) - \theta(x - x')\theta(1 - x)
\end{equation}
In order for Eq.~(\ref{eq:diehlexpr}) to be consistent with the
forward evolution, one should find:
\begin{equation}\label{eq:forwardlimit}
  \lim_{\xi\rightarrow 0}\frac{1}{|2\xi|} V_{\rm NS}^{(0)}\left(\frac{x}{\xi},\frac{x'}{\xi}\right) \mathop{=}^?
  \frac{1}{x'} P_{\rm NS}\left(\frac{x}{x'}\right) = \frac{1}{x'}2C_F\left[\theta\left(\frac{x}{x'}\right)\theta\left(1-\frac{x}{x'}\right)\frac{1+\left(\frac{x}{x'}\right)^2}{1-\left(\frac{x}{x'}\right)}\right]_+\,,
\end{equation}
such that Eq.~(\ref{eq:eveq}) exactly reduces to the DGLAP
equation. However, if one takes the explicit limit for
$\xi\rightarrow 0$ of Eq.~(\ref{eq:diehlexpr}) one finds:\footnote{The
  factor $\theta\left(\frac{x}{x'}\right)$ comes from the factor
  $\theta(-x+x')$ in Eq.~(\ref{eq:supportdiehl}) that can be rewritten
  as
  $\theta\left(\frac{x}{x'}\right)\theta\left(1-\frac{x}{x'}\right)$.}
\begin{equation}\label{eq:forwardlimit2}
\lim_{\xi\rightarrow 0}\frac{1}{|2\xi|} V_{\rm NS}^{(0)}\left(\frac{x}{\xi},\frac{x'}{\xi}\right) = 2C_F\left[\frac{1}{x'}\theta\left(\frac{x}{x'}\right)\left(1-\frac{x}{x'}\right)\frac{1+\left(\frac{x}{x'}\right)^2}{1-\left(\frac{x}{x'}\right)}\right]_+\,.
\end{equation}
Therefore, as compared to Eq.~(\ref{eq:forwardlimit}), the factor
$1/x'$ in Eq.~(\ref{eq:forwardlimit2}) appears \textit{inside} the
+-prescription sign rather than outside which makes the two
expressions effectively different under integration. The difference
amounts to a local term that can be quantified by knowing that:
\begin{equation}
\left[yg(y) \right]_+=y\left[g(y)\right]_+ + \delta(1-y)\int_0^1dz\,(1-z)g(z)\,.
\end{equation}
Notice that, thanks to the factor $(1-z)$, the integral in the
r.h.s. of the above equation converges despite the singularity of
$g$. For example:
\begin{equation}
  \left[\frac{y}{1-y}\right]_+=y\left[\frac1{1-y}\right]_+
  + \delta(1-y)\,.
\end{equation}
Finally, one finds that the forward limit of Eq.~(\ref{eq:diehlexpr}) gives:
\begin{equation}
  \lim_{\xi\rightarrow 0}\frac{1}{|2\xi|} V_{\rm
    NS}^{(0)}\left(\frac{x}{\xi},\frac{x'}{\xi}\right) =
  \frac{1}{x'} \left[P_{\rm NS}\left(\frac{x}{x'}\right)+\frac{4}{3}C_F\delta\left(1-\frac{x}{x'}\right)\right]\,,
\end{equation}
which does \textit{not} reproduce the DGLAP equation due to the
presence of an additional local term.

\subsection{On Vinnikov's code}

The purpose of this section is to draw the attention on a possible
incongruence of the GPD evolution code developed by Vinnikov and
presented in Ref.~\cite{Vinnikov:2006xw}. For definiteness, we
concentrate on the non-singlet $H_{\rm NS}$ GPD in the DGLAP region
$x>\xi$, whose evolution equation is given in Eq.~(29). For
convenience, we report that equation here:
\begin{equation}\label{eq:Vinnikov}
\begin{array}{rcl}
\displaystyle \frac{d H_{\rm NS}(x,\xi,Q^2)}{d\ln
  Q^2}&=&\displaystyle\frac{2\alpha_s(Q^2)}{3\pi}\Bigg[\int_x^1 dy
          \frac{x^2+y^2-2\xi^2}{(y-x)(y^2-\xi^2)}\left(H_{\rm
          NS}(y,\xi,Q^2)-H_{\rm NS}(x,\xi,Q^2)\right)\\
\\
&+&\displaystyle H_{\rm
    NS}(x,\xi,Q^2)\bigg(\frac32+2\ln(1-x)+\frac{x-\xi}{2\xi}\ln((x-\xi)(1+\xi))\\
\\
&-&\displaystyle \frac{x+\xi}{2\xi}\ln((x+\xi)(1-\xi))\bigg)\Bigg]\,,
\end{array}
\end{equation}
and take the forward limit $\xi\rightarrow 0$, obtaining:
\begin{equation}\label{eq:Vinnikov}
\begin{array}{rcl}
\displaystyle \frac{d H_{\rm NS}(x,0,Q^2)}{d\ln
  Q^2}&=&\displaystyle\frac{2\alpha_s(Q^2)}{3\pi}\Bigg[\int_x^1 dy
          \frac{x^2+y^2}{y^2 (y-x)}\left(H_{\rm
          NS}(y,0,Q^2)-H_{\rm NS}(x,0,Q^2)\right)\\
\\
&+&\displaystyle H_{\rm  NS}(x,0,Q^2)\left(\frac32+2\ln(1-x)\right)\Bigg]\,,
\end{array}
\end{equation}

The limit for $\xi\rightarrow 0$ of the equation above should
reproduce the usual DGLAP evolution equation:
\begin{equation}
\displaystyle \frac{d H_{\rm 
NS}(x,0,Q^2)}{d\ln
  Q^2}=\frac{\alpha_s(Q^2)}{4\pi}\int_x^1 \frac{dy}{y}\left[\hat{P}_{\rm
  NS}\left(\frac{x}{y}\right)\right]_+H_{\rm NS}\left(y,0,Q^2\right)\,,
\end{equation}
where:
\begin{equation}
\hat{P}_{\rm NS}\left(z\right)=2C_F\frac{1+z^2}{1-z}\,,
\end{equation}
with $C_F=4/3$. Written explicitly and accounting for the additional
local term arising from the incompleteness of the convolution
integral, one finds:
\begin{equation}\label{eq:DGLAPexpl}
\begin{array}{rcl}
  \displaystyle \frac{d H_{\rm NS}(x,0,Q^2)}{d\ln
    Q^2}&=&\displaystyle \frac{2\alpha_s(Q^2)}{3\pi}\Bigg[\int_x^1
            dy\,\frac{x^2+y^2}{y^3(y-x)}\left(y
            H_{\rm NS}\left(y,0,Q^2\right)-xH_{\rm
      NS}(x,0,Q^2)\right)\\
\\
 &+&\displaystyle H_{\rm NS}(x,0,Q^2)\left(\frac{x(x+2)}{2}+2\ln(1-x)\right)\Bigg]\,,
\end{array}
\end{equation}
which evidently differs from Eq.~(\ref{eq:Vinnikov}). By inspection,
one observes that the difference can be partially traced back to the
issue discussed in Sect.~(\ref{sec:endpoint}). An interesting
observation is that, for $x\rightarrow 1$, the two expressions tend to
coincide. This means that the difference is larger at small values of
$x$. This fact may have concurred to cause the oversight of this
discrepancy in past numerical comparisons.

\subsection{On Ji's evolution equation}

In this section we discuss the evolution equations derived by Ji in
Ref.~\cite{Ji:1996nm}. This form of the evolution equation is dubbed
``near-forward'' in Ref.~\cite{Blumlein:1999sc} because it closely
resembles the DGLAP equation. However, in Ref.~\cite{Ji:1996nm} two
different equations apply to the regions $x<\xi$ and $x>\xi$. In this
section, we will unify them showing that the resulting one-loop
non-singlet off-forward anomalous dimension cannot be written as a
fully $+$-prescribed distribution.

We start by considering Eqs.~(15)-(17) of Ref.~\cite{Ji:1996nm}. The
first step is to replace $\xi/2$ with $\xi$ to match our
notation. Then we consider the subtraction integrals in Eq.~(16)
keeping in mind that they apply to both regions $x<\xi$ and $x>\xi$:\footnote{Note that all
  divergent integrals considered here are implicitly assumed to be
  principal-valued integrals such that:
$$
\int_{-1}^{1}\frac{dt}{t}=0\,.
$$
This allows us to omit the $\pm i\epsilon$ terms.}
\begin{equation}
  \int_{\pm \xi}^x\frac{dy}{y-x} = -\int_{\pm \kappa}^1\frac{dz}{1-z}
  = -\int_{0}^1\frac{dz}{1-z}+\int_{1\mp \kappa}^1\frac{dt}{t}=-\int_{0}^1\frac{dz}{1-z}-\ln(|1\mp\kappa|)\,,
\end{equation}
with:
\begin{equation}\label{eq:kappadef}
  \kappa = \frac{\xi}{x}\,,
\end{equation}
such that the full local term in Eq.~(16) becomes:
\begin{equation}
 \frac{3}{2}+ \int_{\xi}^x\frac{dy}{y-x}+\int_{-\xi}^x\frac{dy}{y-x}
  = \frac{3}{2}-2\int_{0}^1\frac{dz}{1-z}-\ln\left(|1-\kappa^2|\right)\,,
\end{equation}
Considering the symmetry for $\xi\leftrightarrow -\xi$ of the
evolution kernel in Eq.~(17) of Ref.~\cite{Ji:1996nm}, we can write
Eq.~(15) valid for $\kappa<1$ in a more compact way as:
\begin{equation}\label{eq:ji1}
\mu^2\frac{d}{d\mu^2}f^-(x,\xi) = \frac{\alpha_s(\mu)}{4\pi}\int_x^1\frac{dy}{y}\mathcal{P}_1^{-,(0)}(y,\kappa)f^-\left(\frac{x}{y},\xi\right)\,,
\end{equation}
with:
\begin{equation}\label{eq:p1minus0}
\begin{array}{rcl}
\displaystyle \mathcal{P}_1^{-,(0)}(y,\kappa) &=& \displaystyle
                                                  2P_{\rm NS}(y,2\kappa y) + \delta(1-x)2C_F\left(\frac{3}{2}-2\int_{0}^{1}\frac{dy}{1-y}-\ln(|1-\kappa^2|)\right) \\
\\
&=& \displaystyle  2C_F\left\{\left(\frac{2}{1-y}\right)_+-\frac{1
  +y}{1-\kappa^2y^2}+\delta(1-y)\left[\frac32-
  \ln\left(|1-\kappa^2|\right)\right]\right\}\\
\\
&=& \displaystyle  2C_F\left\{\left[\frac{1
  +(1-2\kappa^2)y^2}{(1-y)(1-\kappa^2y^2)}\right]_++\delta(1-y)\left[\frac32+\left(\frac{1}{2\kappa^2}-1\right)
  \ln\left(|1-\kappa^2|\right)+\frac{1}{2\kappa}\ln\left(\left|\frac{1-\kappa}{1+\kappa}\right|\right)\right]\right\}\,,
\end{array}
\end{equation}
where $P_{\rm NS}$ is given in Eq.~(17) of Ref.~\cite{Ji:1996nm}. The
splitting function $\mathcal{P}_1^{-,(0)}$ is such that:
\begin{equation}\label{eq:nonzeroplusGPD}
\int_0^1dy\,\mathcal{P}_1^{-,(0)}(y,\kappa) =2C_F\left[\frac32+\left(\frac{1}{2k^2}-1\right) \ln\left(|1-\kappa^2|\right)+\frac{1}{2\kappa}\ln\left(\left|\frac{1-\kappa}{1+\kappa}\right|\right)\right]\,,
\end{equation}
which means that it cannot be written as a fully $+$-prescribed
distribution. However, the integral above correctly tends to zero as
$\kappa\rightarrow 0$ allowing one to recover the usual DGLAP
splitting function in the forward limit:
\begin{equation}\label{eq:dglapsplitting}
  \lim_{\kappa\rightarrow 0}\mathcal{P}_1^{-,(0)}(y,\kappa) = 2C_F\left[\frac{1+y^2}{1-y}\right]_+\,.
\end{equation}

It should also be pointed out that also the limit for
$\kappa\rightarrow 1$ of Eq.~(\ref{eq:p1minus0}) is well-behaved:
\begin{equation}
\lim_{\kappa\rightarrow 1}\mathcal{P}_1^{-,(0)}(y,\kappa) = 2C_F\left\{\left[\frac{1}{1-y}\right]_++\delta(1-y)\left[\frac32-\ln(2)\right]\right\}\,.
\end{equation}
which is necessary to have a smooth transition of the GPDs from the
DGLAP ($x>\xi$) to the ERBL ($x<\xi$) region.

We now consider Eqs.~(18) and~(19) of
Ref.~\cite{Ji:1996nm} valid for $\kappa>1$. Interestingly, after
some algebra, we find:
\begin{equation}\label{eq:ji2}
\mu^2\frac{d}{d\mu^2}f^-(x,\xi) = \frac{\alpha_s(\mu)}{4\pi}\left[\int_x^1\frac{dy}{y}\mathcal{P}_1^{-,(0)}(y,\kappa)f^-\left(\frac{x}{y},\xi\right)+\int_x^\infty\frac{dy}{y}\mathcal{P}_2^{-,(0)}(y,\kappa)f^-\left(\frac{x}{y},\xi\right)\right]\,,
\end{equation}
with $\mathcal{P}_1^{-,(0)}$ given by:
\begin{equation}\label{eq:p1minus02}
\displaystyle \mathcal{P}_1^{-,(0)}(y,\kappa)=2P_{\rm NS}'(y,2\kappa y) +2P_{\rm NS}'(y,-2\kappa y)+ \delta(1-x)2C_F\left(\frac{3}{2}-2\int_{0}^{1}\frac{dy}{1-y}-\ln(|1-\kappa^2|)\right)\,,
\end{equation}
with $P_{\rm NS}'$ is given in Eq.~(19) of Ref.~\cite{Ji:1996nm} and
remarkably equal to the expression in Eq.~(\ref{eq:p1minus0})
signifying that:
\begin{equation}
  P_{\rm NS}(y,2\kappa y) =  P_{\rm NS}'(y,2\kappa y) +P_{\rm NS}'(y,-2\kappa y)\,.
\end{equation}
While:
\begin{equation}\label{eq:p2minus0}
  \mathcal{P}_2^{-,(0)}(y,\kappa) = -2P_{\rm NS}'(y,-2\kappa y) +2P_{\rm
    NS}'(-y,2\kappa y) = 2C_F(\kappa-1)\frac{y+(1+2\kappa)y^3}{(1-y^2)(1-\kappa^2y^2)}\,.
\end{equation}
It is very interesting to notice that $\mathcal{P}_2^{-,(0)}$ is
proportional to $(\kappa-1)$ that finally guarantees the continuity of
GPDs at $k=1$.

We observe that, within the integration interval, the splitting
function $\mathcal{P}_2^{-,(0)}$ is singular at $y=1$.\footnote{The
  singularities at $y=-1$ and $y=\pm1/\kappa$ are all placed below
  $y=x$ that is the lower integration bound and thus do not cause any
  problem.} However, as pointed out above, the second integral on the
r.h.s. of Eq.~(\ref{eq:ji2}) has to be regarded as principal-valued
therefore it is well-defined. In order to treat this integral
numerically we consider the specific case:
\begin{equation}
I=\int_x^\infty dy\,\frac{f(y)}{1-y}\,,
\end{equation}
where $f$ is a test function well-behaved over the full integration
range. If one subtracts and adds back the divergence at $y=1$,
\textit{i.e.}:
\begin{equation}
f(1)\int_0^1dy\,g(y)\,,
\end{equation}
one can rearrange the integral as follows:
\begin{equation}\label{eq:plusplusdist}
I=\int_x^\infty\frac{dy}{1-y}\left[f(y)-f(1)\left(1+\theta(y-1)\frac{1-y}{y}\right)\right]+f(1)\ln(1-x)\equiv
\int_x^\infty dy\left(\frac{1}{1-y}\right)_{++}f(y)\,,
\end{equation}
which effectively defines the $++$-distribution. It should be noticed
that this definition is specific to the function $1/(1-y)$. In case of
a different singular function the function that multiplies
$\theta(y-1)$ would be different. The advantage of this rearrangement
is that the integrand is free of the divergence at $y=1$ and is thus
amenable to numerical integration. Also, the $++$-distribution reduces
to the standard $+$-distribution when the upper integration bound is
one rather than infinity. In this sense the $++$-distribution
generalises the $+$-distribution to ERBL-like integrals.

In view of the use of Eq.~(\ref{eq:plusplusdist}), it is convenient to
rewrite Eq.~(\ref{eq:p2minus0}) as follows:
\begin{equation}\label{eq:p2minus01}
  \mathcal{P}_2^{-,(0)}(y,\kappa) =
  2C_F\left[\frac{1+(1+\kappa)y+(1+\kappa -\kappa^2)y^2}{(1+y)(1-\kappa^2y^2)}-\left(\frac{1}{1-y}\right)_{++}\right]\,,
\end{equation}
where the first term in the squared bracket is regular at $y=1$.

Finally, Eqs.~(\ref{eq:ji1}) and Eq.~(\ref{eq:ji2}) can be combined as
follows:
\begin{equation}\label{eq:jitot}
  \mu^2\frac{d}{d\mu^2}f^-(x,\xi) = \frac{\alpha_s(\mu)}{4\pi}\left[\int_x^1\frac{dy}{y}\mathcal{P}_1^{-,(0)}(y,\kappa)f^-\left(\frac{x}{y},\xi\right)+\theta(\kappa-1)\int_x^\infty\frac{dy}{y}\mathcal{P}_2^{-,(0)}(y,\kappa)f^-\left(\frac{x}{y},\xi\right)\right]\,,
\end{equation}
to obtain a single DGLAP-like evolution equation valid for all values
of $\kappa$.

In the limit $\kappa\rightarrow 0$, the second integral in the
r.h.s. of Eq.~(\ref{eq:jitot}) drops and the splitting function
$\mathcal{P}_1^{-,(0)}$ reduces to the one-loop non-singlet DGLAP
splitting function (see Eq.~(\ref{eq:dglapsplitting})) so that, as
expected, Eq.~(\ref{eq:jitot}) becomes the DGLAP equation in this
limit.

It is also interesting to verify that also the ERBL equation is
recovered in the limit $\xi\rightarrow 1$. Given the definition of
$\kappa$, Eq.~(\ref{eq:kappadef}), this limit is attained by taking
$\kappa \rightarrow 1/x$. However, the limit procedure is more subtle
than in the DGLAP case due to the presence of $+$-prescriptions and
explicit local terms that need to cooperate to give the right result.

We make use of Eqs.~(\ref{eq:p1minus02}) and~(\ref{eq:p2minus0}) to
write the evolution equation in terms of the function $P_{\rm NS}'$ in
a form similar to that originally given in Ref.~\cite{Ji:1996nm} but
more compactly as:
\begin{equation}\label{eq:erbl2}
  \mu^2\frac{d}{d\mu^2}f^-(x,1) =
  \frac{\alpha_s(\mu)}{4\pi}\left[\int_{-1}^1 dy\,V_{\rm NS}^{(0)}(x,y)f^-\left(y,1\right)\right] \,.
\end{equation}
with:
\begin{equation}
\begin{array}{rcl}
V_{\rm NS}^{(0)}(x,y) &=&\displaystyle \theta(y-x)\left[\frac{2}{y}P_{\rm
    NS}'\left(\frac{x}{y},\frac{2}{y}\right)
                          \right]-2C_F\delta\left(y-x\right)\int_{-1}^1 dz \,\frac{\theta(z-x)}{z-x}\\
\\
&+&\displaystyle \theta(x-y)\left[-\frac{2}{y}P_{\rm
    NS}'\left(\frac{x}{y},-\frac{2}{y}\right)
                          \right]+2C_F\delta\left(x-y\right)\int_{-1}^{1}dz\,\frac{\theta(x-z)}{z-x}\\
\\
&+&\displaystyle 3C_F\delta\left(y-x\right)\,,
\end{array}
\end{equation}
where:
\begin{equation}
\frac{2}{y}P_{\rm NS}'\left(\frac{x}{y},\frac{2}{y}\right)=2C_F\frac{1+x}{1+y}\left(\frac{1}{2}+\frac{1}{y-x}\right)\,.
\end{equation}
In order to make a step towards the ERBL equation, we change the
variables $x$ and $y$ with:
\begin{equation}
\begin{array}{l}
\displaystyle t= \frac12\left(x + 1\right)\,,\\
\\
\displaystyle u = \frac12\left(y + 1\right)\,,\\
\end{array}
\end{equation}
such that the evolution variable becomes:
\begin{equation}\label{eq:erbl2}
  \mu^2\frac{d}{d\mu^2}\Phi^-(t) = \frac{\alpha_s(\mu)}{4\pi}\left[\int_{0}^1 du\,\overline{V}_{\rm NS}^{(0)}(t,u) \Phi^-\left(u\right)\right] \,.
\end{equation}
with $\Phi^-(t) = f^-(x,1)$ and:
\begin{equation}\label{eq:anomdim}
\begin{array}{rcl}
\overline{V}_{\rm NS}^{(0)}(t,u) &=&\displaystyle
2C_F\bigg[\theta(u-t)\left(\frac{t-1}{u}+\frac1{u-t}-\delta(u-t)\int_0^1 du'\frac{\theta(u'-t)}{u'-t}\right)\\
\\
&-&\displaystyle \theta(t-u)\left(\frac{t}{1-u}+\frac1{u-t}-\delta(t-u)\int_0^1 du'\frac{\theta(t-u')}{u'-t}\right) +\frac32\delta\left(u-t\right)\bigg]\,.
\end{array}
\end{equation}
Now we define:
\begin{equation}
  \left[f(t,u)\right]_+\equiv f(t,u)-\delta(u-t)\int_0^1du'\,f(t,u') \,,
\end{equation}
where $f$ has a single pole at $u=t$, so that we can write
Eq.~(\ref{eq:anomdim}) more compactly as:
\begin{equation}\label{eq:anomdim1}
  \overline{V}_{\rm NS}^{(0)}(t,u) =
  2C_F\left\{\left[\theta(u-t)\frac{t-1}{u}+\left(\frac{\theta(u-t)}{u-t}\right)_+\right]-\left[\theta(t-u)\frac{t}{1-u}+\left(\frac{\theta(t-u)}{u-t}\right)_+\right]+\frac32\delta\left(u-t\right)\right\}\,.
\end{equation}
This confirms the result of Ref.~\cite{Blumlein:1999sc} modulo the
fact that, for achieving a correct cancellation of the divergencies,
the $\theta$-function for the $+$-prescribed terms needs to be inside
the $+$-prescription sign itself rather than outside.

One can check that integrating $\overline{V}_{\rm NS}^{(0)}$ over $t$
gives zero:\footnote{Note that the two $+$-prescribed terms when
  integrated over $t$ do not individually give zero but their
  combination does.}
\begin{equation}\label{eq:nonzeroplusERBL}
  \int_0^1dt\,\overline{V}_{\rm NS}^{(0)}(t,u) = 0\,.
\end{equation}
This finally confirms that $\overline{V}_{\rm NS}^{(0)}$ as derived
from Ref.~\cite{Ji:1996nm} admits a fully $+$-prescribed form, that
is:
\begin{equation}\label{eq:anomdim1}
  \overline{V}_{\rm NS}^{(0)}(t,u) =
  2C_F\left\{\theta(u-t)\left[\frac{t-1}{u}+\frac{1}{u-t}\right]-\theta(t-u)\left[\frac{t}{1-u}+\frac{1}{u-t}\right]\right\}_+\,.
\end{equation}
This was also explicitly derived in Ref.~\cite{Mikhailov:1984ii} and
argued that this property must hold for symmetry reasons.

It is now interesting to derive an ERBL-like evolution equation for
GPDs at one loop. Assuming $\xi>0$, this equation can be written as:
\begin{equation}
\frac{d}{d\ln\mu^2} f^{-}(x,\xi)=
\frac{\alpha_s(\mu)}{4\pi}\int_{-1}^1\frac{dy}{\xi} \mathbb{V}_{\rm NS}^{(0)}\left(\frac{x}{\xi},\frac{y}{\xi}\right) f^{-}(y,\xi)\,.
\end{equation}
A simple change of variables allows one to write this equation as:
\begin{equation}
\frac{d}{d\ln\mu^2} f^{-}(\xi x,\xi)=
\frac{\alpha_s(\mu)}{4\pi}\int_{-1/\xi}^{1/\xi}dy\, \mathbb{V}_{\rm NS}^{(0)}\left(x,y\right) f^{-}(\xi y,\xi)\,,
\end{equation}
where the evolution kernel is given by:
\begin{equation}
\begin{array}{rcl}
\mathbb{V}_{\rm NS}^{(0)}\left(x,y\right) &=& \displaystyle
                                              \theta(1-|x|){V}_{\rm
                                              NS}^{(0)}\left(x,y\right)\\
\\
&+&\displaystyle
    \theta(|x|-1) 2C_F\Bigg\{\left[\theta(y-|x|)\left(\frac{2}{y-|x|}+\frac{|x|+y}{1-y^2}\right)\right]_+\\
\\
&+&\displaystyle\delta(|x|-y)\bigg[\frac{3}{2}-\ln\left(\frac{|x|-1}{|x|+1}\right)\\
\\
&+&\displaystyle 2 \ln\left(\frac{1-\xi |x|}{\xi(|x|-1)}\right)\left[\theta\left(\frac{1+\xi}{2\xi}-|x|\right)-\theta\left(|x|-\frac{1+\xi}{2\xi}\right)\right]\\
\\
&+&\displaystyle \frac12(|x|+1)\ln\left(\frac{1-\xi}{\xi(|x|-1)}\right)-\frac12(|x|-1)\ln\left(\frac{1+\xi}{\xi(|x|+1)}\right) \bigg]\Bigg\}\,,
\end{array}
\end{equation}
with ${V}_{\rm NS}^{(0)}$ given in Eq.~(\ref{eq:anomdim}) and where we
have exploited the fact that $f^-(x,\xi)=f^-(-x,\xi)$. In addition,
the $+$-prescription in the second line of the above equation has the
following \textit{ad hoc} definition:
\begin{equation}
\left[f(x,y)\right]_+=f(x,y)-\delta(x-y)\int_1^{1/\xi}dy\,f(x,y)\,.
\end{equation}

A question arises: does the fact that the GPD anomalous dimension
(\textit{cfr.}  Eq.~(\ref{eq:nonzeroplusGPD})) does not admit a fully
$+$-prescribed form violate any conservation law? To answer this
question, we notice that the fact that the non-singlet DGLAP anomalous
dimension integrates to zero (see Eq.~(\ref{eq:dglapsplitting})), and
thus admits a $+$-prescribed form, derives from the conservation of
the total number of quarks minus anti-quarks (valence sum rule):
\begin{equation}
\int_0^1dx\, f^-(x,0) = \mbox{constant}\,.
\end{equation}
Taking the derivative of this equation w.r.t. $\ln\mu^2$ and using the
DGLAP equation gives:
\begin{equation}\label{eq:valsumrule}
\begin{array}{rcl}
0&=&\displaystyle\int_0^1dx\int_x^1\frac{dy}{y}
     \mathcal{P}_1^-(y,0)f^-\left(\frac{x}{y},0\right) =
     \int_0^1dy\,
     \mathcal{P}_1^-(y,0)\int_0^y\frac{dx}{y}\,f^-\left(\frac{x}{y},0\right)\\
\\
&=&\displaystyle \int_0^1dy\,
     \mathcal{P}_1^-(y,0)\int_0^1dz\,f^-\left(z,0\right) =
    \mbox{constant}\times\int_0^1dy\,\mathcal{P}_1^-(y,0)\quad\Leftrightarrow
    \quad \int_0^1dy\,\mathcal{P}_1^-(y,0) = 0\,.
\end{array}
\end{equation}
This clearly justifies the requirement for $\mathcal{P}_1^-(y,0)$ to
be fully $+$-prescribed.

One may try to apply the same argument to GPDs. In this case the
valence sum rule generalises in:
\begin{equation}\label{eq:sumrule}
\int_0^1 dx\,f^-(x,\xi) = F\,,
\end{equation}
where $F$ is independent of $\mu$ and $\xi$ but may (and does) depend
on the momentum transfer $t$. $F$ is usually referred to as form
factor. One should now take the derivative w.r.t. $\ln\mu^2$ and use
Eq.~(\ref{eq:jitot}) but in doing this one needs to take into account
that $\kappa=\xi/x$:
\begin{equation}
\begin{array}{rcl}
  0&=&\displaystyle \int_0^1
       dx\int_0^1\frac{dy}{y}\left[\theta(y-x)\mathcal{P}_1^{-,(0)}\left(\frac{x}{y},\frac{\xi}{x}\right)+\theta(\xi-x)
       \mathcal{P}_2^{-,(0)}\left(\frac{x}{y},\frac{\xi}{x}\right)\right]f^-\left(y,\xi\right)\\
\\
&=&\displaystyle \int_0^1 dy\,f^-\left(y,\xi\right) \left[\int_0^y\frac{dx}{y}\,\mathcal{P}_1^{-,(0)}\left(\frac{x}{y},\frac{\xi}{x}\right)+
       \int_0^\xi \frac{dx}{y}\, \mathcal{P}_2^{-,(0)}\left(\frac{x}{y},\frac{\xi}{x}\right)\right]\\
\\
&=&\displaystyle \int_0^1 dy\,f^-\left(y,\xi\right) \left[\int_0^1dz\,\mathcal{P}_1^{-,(0)}\left(z,\frac{\xi}{yz}\right)+
       \int_0^{\xi/y} dz\, \mathcal{P}_2^{-,(0)}\left(z,\frac{\xi}{yz}\right)\right]\,,
\end{array}
\end{equation}
In order for this relation to be identically true, it is necessary
that:
\begin{equation}\label{eq:valsumruleGPD}
  \int_0^1dz\,\mathcal{P}_1^{-,(0)}\left(z,\frac{\xi}{yz}\right)+
  \int_0^{\xi/y} dz\, \mathcal{P}_2^{-,(0)}\left(z,\frac{\xi}{yz}\right)=0\,.
\end{equation}
Notice that for $\xi\rightarrow 0$, the equality above reduces to
Eq.~(\ref{eq:valsumrule}). It is interesting to verify
Eq.~(\ref{eq:valsumruleGPD}) plugging in the explicit expressions for
$\mathcal{P}_1^{-,(0)}$, Eq.~(\ref{eq:p1minus0}), and
$\mathcal{P}_2^{-,(0)}$, Eq.~(\ref{eq:p2minus0}). One finds:
\begin{equation}
  \int_0^1dz\,\mathcal{P}_1^{-,(0)}\left(z,\frac{\xi}{yz}\right)=2C_F\left[-\frac{3}{2}\frac{\xi^2}{\xi^2-y^2}-\ln\left(\left|1-\frac{\xi^2}{y^2}\right|\right)\right]\,,
\end{equation}
that correctly tends to zero as $\xi\rightarrow 0$, and:
\begin{equation}
  \int_0^{\xi/y} dz\, \mathcal{P}_2^{-,(0)}\left(z,\frac{\xi}{yz}\right)=2C_F\left[\frac{3}{2}\frac{\xi^2}{\xi^2-y^2}+\ln\left(\left|1-\frac{\xi^2}{y^2}\right|\right)\right]\,,
\end{equation}
such that Eq.~(\ref{eq:valsumruleGPD}) is fulfilled. Despite
Eq.~(\ref{eq:valsumruleGPD}) has been explicitly proved at one-loop,
the same relation must hold order by order in perturbation theory.

It is important to notice that the constraint on the non-singlet GPD
anomalous dimensions deriving from the valence sum rule, and resulting
in Eq.~(\ref{eq:valsumruleGPD}), does not take the form of a
$+$-prescription, Eq.~(\ref{eq:pludistributionn}). A further proof can
be given by considering the non-singlet GPD evolution equation given
in Eq.~(\ref{eq:eveq}) (see also Eq.~(99) of
Ref.~\cite{Diehl:2003ny}). The independence of the form factor from
$\mu$ immediately leads to:
\begin{equation}
\int_{-1}^1dx'
f(x',\xi)\int_{-1}^1dx\left[\hat{V}_{\rm NS}\left(\frac{x}{\xi},\frac{x'}{\xi}\right)\right]_+ = 0\,,
\end{equation}
where $\hat{V}_{\rm NS}$ is nothing but $V _{\rm NS}$ stripped of the supposedly global
$+$-prescription. A simple change of variables (assuming $\xi$
positive) gives:
\begin{equation}
  \int_{-1/\xi}^{1/\xi}dy'
  f(\xi y',\xi)\int_{-1/\xi}^{1/\xi}dy\left[\hat{V}_{\rm NS}\left(y,y'\right)\right]_+ = 0\,.
\end{equation}
Given the fact that the bounds of the inner integral are not $-1$ and
$1$, the effect of the $+$-prescription as given in
Eq.~(\ref{eq:pludistributionn}) cannot give zero. This prevents the
above equation to be identically fulfilled violating polynomiality of
GPDs. We can thus conclude that $V _{\rm NS}$ \textit{cannot} be
written as a fully $+$-prescribed function.

Having ascertained that the evolution equation from
Ref.~\cite{Ji:1996nm} is well-behaved for the non-singlet distribution
$f^-$, we move on to consider the singlet $f^+$ and the gluon $f_g$
distributions. As in the standard DGLAP evolution equation, singlet
and gluon GPDs couple under evolution. Defining $f^+$ as the column
vector of singlet and gluon GPDs, the corresponding anomalous
dimension $\mathcal{P}^+$ is a matrix in flavour space:
\begin{equation}\label{eq:singmat}
  \mathcal{P}^+=\begin{pmatrix}
    \mathcal{P}_{\rm SS}& \mathcal{P}_{\rm SG}\\
    \mathcal{P}_{\rm GS}& \mathcal{P}_{\rm GG}
\end{pmatrix}\,.
\end{equation}
Following the same procedure discussed above for the non-singlet
distribution $f^-$, the one-loop evolution equation for $f^+$ reads:
\begin{equation}\label{eq:jitotsing}
  \mu^2\frac{d}{d\mu^2}f^+(x,\xi) = \frac{\alpha_s(\mu)}{4\pi}\left[\int_x^1\frac{dy}{y}\mathcal{P}_1^{+,(0)}(y,\kappa)f^-\left(\frac{x}{y},\xi\right)+\theta(\kappa-1)\int_x^\infty\frac{dy}{y}\mathcal{P}_2^{+,(0)}(y,\kappa)f^+\left(\frac{x}{y},\xi\right)\right]\,.
\end{equation}
The single splitting functions $\mathcal{P}_{1}$ and $\mathcal{P}_2$ are
derived from the expression Ref.~\cite{Ji:1996nm} as:
\begin{equation}
\begin{array}{rcl}
\mathcal{P}_{I,1}(y,\kappa) &=& 2P_I(y,\kappa y)= 2P_I'(y,\kappa y) +
                                2P_I'(y,-\kappa y)\,,\\
\\
\mathcal{P}_{I,2}(y,\kappa) &=& - 2P_I'(y,-\kappa y) - 2P_I'(-y,\kappa y)\,,
\end{array}
\end{equation}
with $I=\rm SS,SG,GS,SS$. This leads to:
\begin{equation}
\left\{\begin{array}{rcl}
\mathcal{P}_{\rm SS, 1}^{(0)}(y,\kappa) &=& \mathcal{P}_{1}^{-,(0)}(y,\kappa)\,,\\
\\
\mathcal{P}_{\rm SS, 2}^{(0)}(y,\kappa) &=& \displaystyle 2C_F(1-\kappa)\left[\frac{1-(1+2\kappa+2\kappa^2)y^2}{\kappa(1-y^2)(1-\kappa^2y^2)}\right]=2C_F\left[\frac{(1+\kappa)(1+y)+\kappa^3y^2}{(1+y)(1-\kappa^2y^2)}-\left(\frac{1}{1-y}\right)_{++}\right]\,,
\end{array}\right.
\end{equation}

\begin{equation}
\left\{\begin{array}{rcl}
\mathcal{P}_{\rm SG, 1}^{(0)}(y,\kappa) &=& \displaystyle 4n_f T_R\left[\frac{y^2+(1-y)^2-\kappa^2y^2}{(1-\kappa^2y^2)^2}\right]\,,\\
\\
\mathcal{P}_{\rm SG, 2}^{(0)}(y,\kappa) &=& \displaystyle 4n_f T_R(1-\kappa)\left[\frac{1-\kappa(\kappa+2)y^2}{\kappa(1-\kappa^2y^2)^2}\right]\,,
\end{array}\right.
\end{equation}

\begin{equation}
\left\{\begin{array}{rcl}
  \mathcal{P}_{\rm GS, 1}^{(0)}(y,\kappa) &=& \displaystyle 2C_F\left[\frac{1+(1-y)^2-\kappa^2y^2}{y(1-\kappa^2y^2)}\right]\,,\\
  \\
  \mathcal{P}_{\rm GS, 2}^{(0)}(y,\kappa) &=& \displaystyle
                                          2C_F(1-\kappa)\left[\frac{2-\kappa(\kappa+1)y^2}{\kappa
                                          y(1-\kappa^2 y^2)}\right]\,,
\end{array}\right.
\end{equation}

\begin{equation}
\left\{\begin{array}{rcl}
  \mathcal{P}_{\rm GG, 1}^{(0)}(y,\kappa) &=& \displaystyle
                                          \frac{4C_A}{(1-\kappa^2
                                          y^2)^2}\left[(1-\kappa^2) (1-\kappa^2y^2)\frac{y}{(1-y)_+}+\frac{1-y}{y}+y(1-y)\right]+\delta(1-y)\left(\frac{11C_A-4T_R n_f}{3}\right)\,,\\
  \\
  \mathcal{P}_{\rm GG, 2}^{(0)}(y,\kappa) &=& \displaystyle 2C_A\frac{1-\kappa^2}{1-\kappa^2y^2}\left[\frac{2(1+y^2)}{(1+\kappa)(1-\kappa^2y^2)}-\frac{1}{\kappa}+2-\frac{1}{1+y}-\left(\frac{1}{1-y}\right)_{++}\right]\,.
\end{array}\right.
\end{equation}
In all cases, the limit for $\kappa\rightarrow 0$ of $\mathcal{P}_1$
reproduces the one-loop DGLAP splitting functions. In addition, we
also notice that all $\mathcal{P}_2$ are proportional to
$\kappa-1$. Along with the fact that all $\mathcal{P}_1$ are
well-behaved at $\kappa=1$, this guarantees the continuity of GPDs at
the cross-over point $x=\xi$.

We now compute the ERBL limit by taking $\kappa\rightarrow 1/x$. To do
so, we use the ERBL-compliant form of the evolution equation:
\begin{equation}\label{eq:erblsing}
  \mu^2\frac{d}{d\mu^2}f^+(x,1) =
  \frac{\alpha_s(\mu)}{4\pi}\left[\int_{-1}^1 dy\,\mathcal{V}^{(0)}(x,y)f^+\left(y,1\right)\right] \,,
\end{equation}
where $\mathcal{V}^{(0)}$ is a matrix in flavour space with the same
structure of $\mathcal{P}^+$ in Eq.~(\ref{eq:singmat}), whose
components can be written in terms of the $P_{I}'$ functions as follows:
\begin{equation}
\begin{array}{rcl}
V_{\rm SS}^{(0)}(x,y) &=&\displaystyle \theta(y-x)\left[\frac{2}{y}P_{\rm
    SS}'\left(\frac{x}{y},\frac{2}{y}\right)
                          \right]-2C_F\delta\left(y-x\right)\int_{-1}^1 dz \,\frac{\theta(z-x)}{z-x}\\
\\
&+&\displaystyle \theta(x-y)\left[-\frac{2}{y}P_{\rm
    SS}'\left(\frac{x}{y},-\frac{2}{y}\right)
                          \right]+2C_F\delta\left(x-y\right)\int_{-1}^{1}dz\,\frac{\theta(x-z)}{z-x}\\
\\
&+&\displaystyle 3C_F\delta\left(y-x\right)\,, \\
\\
V_{\rm SG,GS}^{(0)}(x,y) &=&\displaystyle \theta(y-x)\left[\frac{2}{y}P_{\rm
    SG,GS}'\left(\frac{x}{y},\frac{2}{y}\right)
                          \right]+\displaystyle \theta(x-y)\left[-\frac{2}{y}P_{\rm
    SG,GS}'\left(\frac{x}{y},-\frac{2}{y}\right)
                          \right]\,,\\
\\
V_{\rm GG}^{(0)}(x,y) &=&\displaystyle \theta(y-x)\left[\frac{2}{y}P_{\rm
    GG}'\left(\frac{x}{y},\frac{2}{y}\right)
                          \right]-2C_A\delta\left(y-x\right)\int_{-1}^1 dz \,\frac{\theta(z-x)}{z-x}\\
\\
&+&\displaystyle \theta(x-y)\left[-\frac{2}{y}P_{\rm
    GG}'\left(\frac{x}{y},-\frac{2}{y}\right)
                          \right]+2C_A\delta\left(x-y\right)\int_{-1}^{1}dz\,\frac{\theta(x-z)}{z-x}\\
\\
&+&\displaystyle \left(\frac{11C_A-4T_R
    n_f}{3}\right)\delta\left(y-x\right)\,.
\end{array}
\end{equation}
The explicit expressions read:
\begin{equation}
\begin{array}{rcl}
V_{\rm SS}^{(0)}(x,y) &=&\displaystyle V_{\rm NS}^{(0)}(x,y)\,,\\
\\
V_{\rm SG}^{(0)}(x,y) &=&\displaystyle \,,\\
\\
V_{\rm GS}^{(0)}(x,y) &=&\displaystyle \,,\\
\\
V_{\rm GG}^{(0)}(x,y) &=&\displaystyle \theta(y-x)\left[\frac{2}{y}P_{\rm
    GG}'\left(\frac{x}{y},\frac{2}{y}\right)
                          \right]-2C_A\delta\left(y-x\right)\int_{-1}^1 dz \,\frac{\theta(z-x)}{z-x}\\
\\
&+&\displaystyle \theta(x-y)\left[-\frac{2}{y}P_{\rm
    GG}'\left(\frac{x}{y},-\frac{2}{y}\right)
                          \right]+2C_A\delta\left(x-y\right)\int_{-1}^{1}dz\,\frac{\theta(x-z)}{z-x}\\
\\
&+&\displaystyle \left(\frac{11C_A-4T_R
    n_f}{3}\right)\delta\left(y-x\right)\,.
\end{array}
\end{equation}

\newpage

\begin{thebibliography}{alp}

%\cite{Diehl:2003ny}
\bibitem{Diehl:2003ny}
  M.~Diehl,
  %``Generalized parton distributions,''
  Phys.\ Rept.\  {\bf 388} (2003) 41
  doi:10.1016/j.physrep.2003.08.002, 10.3204/DESY-THESIS-2003-018
  [hep-ph/0307382].
  %%CITATION = doi:10.1016/j.physrep.2003.08.002, 10.3204/DESY-THESIS-2003-018;%%
  %1016 citations counted in INSPIRE as of 30 Oct 2019

%\cite{Blumlein:1999sc}
\bibitem{Blumlein:1999sc}
  J.~Blumlein, B.~Geyer and D.~Robaschik,
  %``The Virtual Compton amplitude in the generalized Bjorken region: twist-2 contributions,''
  Nucl.\ Phys.\ B {\bf 560} (1999) 283
  doi:10.1016/S0550-3213(99)00418-6
  [hep-ph/9903520].
  %%CITATION = doi:10.1016/S0550-3213(99)00418-6;%%
  %86 citations counted in INSPIRE as of 13 Feb 2020

%\cite{Radyushkin:1997ki}
\bibitem{Radyushkin:1997ki}
A.~V.~Radyushkin,
%``Nonforward parton distributions,''
Phys. Rev. D \textbf{56} (1997), 5524-5557
doi:10.1103/PhysRevD.56.5524
[arXiv:hep-ph/9704207 [hep-ph]].
%1104 citations counted in INSPIRE as of 02 Jul 2020

%\cite{Ji:1996nm}
\bibitem{Ji:1996nm}
X.~D.~Ji,
%``Deeply virtual Compton scattering,''
Phys. Rev. D \textbf{55} (1997), 7114-7125
doi:10.1103/PhysRevD.55.7114
[arXiv:hep-ph/9609381 [hep-ph]].
%1183 citations counted in INSPIRE as of 08 Jul 2020

%\cite{Vinnikov:2006xw}
\bibitem{Vinnikov:2006xw}
A.~V.~Vinnikov,
%``Code for prompt numerical computation of the leading order GPD evolution,''
[arXiv:hep-ph/0604248 [hep-ph]].
%21 citations counted in INSPIRE as of 20 Jul 2020

%\cite{Mueller:1998fv}
\bibitem{Mueller:1998fv}
D.~Müller, D.~Robaschik, B.~Geyer, F.~M.~Dittes and J.~Horejsi,
%``Wave functions, evolution equations and evolution kernels from light ray operators of QCD,''
Fortsch. Phys. \textbf{42} (1994), 101-141
doi:10.1002/prop.2190420202
[arXiv:hep-ph/9812448 [hep-ph]].
%1185 citations counted in INSPIRE as of 30 Aug 2020

%\cite{Mikhailov:1984ii}
\bibitem{Mikhailov:1984ii}
S.~V.~Mikhailov and A.~V.~Radyushkin,
%``Evolution Kernels in {QCD}: Two Loop Calculation in Feynman Gauge,''
Nucl. Phys. B \textbf{254} (1985), 89-126
doi:10.1016/0550-3213(85)90213-5
%96 citations counted in INSPIRE as of 30 Aug 2020

\end{thebibliography}

\end{document}
